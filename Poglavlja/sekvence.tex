Dijagrami sekvenci prikazuju tok interakcije između korisnika, sistema i ostalih aktera kroz vremensku osu. Prikazani su ključni procesi: odobravanje registracije, rezervacija ture, dodela vodiča i završne aktivnosti nakon ture. Time se jasno uočava redosled poruka, tačke potvrde i zavisnosti između komponenti, kao i mesta gde se primenjuju pravila validacije i obaveštenja.

Sekvence su posebno važne jer otkrivaju gde je neophodna sinhronizacija, gde se zahteva potvrda, kao i koje akcije pokreću notifikacije i promene statusa. Na osnovu ovih dijagrama moguće je preciznije definisati pozive programskog interfejsa, transakcione granice i očekivane reakcije sistema.

\subsection*{Odobravanje registracije}
Proces obuhvata kreiranje naloga, proveru podataka i administrativno odobravanje za vodiče i operatere. Nakon odobrenja korisnik dobija pristup sistemu, a u slučaju odbijanja dobija objašnjenje razloga i instrukcije za ponovni pokušaj.

\begin{itemize}
    \item Osnovni tok: registracija -> validacija -> čuvanje naloga -> odobravanje.
    \item Alternativni tok: neispravni podaci -> korekcija; odbijanje -> obaveštenje.
\end{itemize}

\begin{figure}[H]
    \centering
    \includegraphics[width=0.95\textwidth]{Slike/seq_registration_approval.png}
    \caption{Dijagram sekvenci: Odobravanje registracije}
    \label{fig:seq_registration_approval}
\end{figure}

\subsection*{Rezervacija ture}
Korisnik bira turu, sistem proverava dostupnost i kreira rezervaciju. Operater i vodič dobijaju obaveštenje o novoj rezervaciji, a korisnik potvrdu. U slučaju popunjenosti, sistem vraća informaciju o nedostupnosti i predlaže alternativne termine.

\begin{itemize}
    \item Osnovni tok: izbor ture -> provera kapaciteta -> kreiranje rezervacije -> potvrda.
    \item Alternativni tok: puna tura -> odbijanje -> predlog drugih termina.
\end{itemize}

\begin{figure}[H]
    \centering
    \includegraphics[width=0.95\textwidth]{Slike/seq_tour_booking.png}
    \caption{Dijagram sekvenci: Rezervacija ture}
    \label{fig:seq_tour_booking}
\end{figure}

\subsection*{Upravljanje turama i dodela vodiča}
Operater kreira turu i dodeljuje vodiča, dok vodič potvrđuje ili odbija dodelu. Sistem evidentira status i obaveštava relevantne strane. Dodela je ograničena raspoloživošću vodiča i vremenskim okvirom ture.

\begin{itemize}
    \item Osnovni tok: kreiranje ture -> dodela vodiča -> potvrda dodele.
    \item Alternativni tok: odbijanje dodele -> izbor drugog vodiča.
\end{itemize}

\begin{figure}[H]
    \centering
    \includegraphics[width=0.95\textwidth]{Slike/seq_tour_management_assignment.png}
    \caption{Dijagram sekvenci: Upravljanje turama i dodela vodiča}
    \label{fig:seq_tour_management_assignment}
\end{figure}

\subsection*{Aktivnosti nakon ture}
Po završetku ture korisnik ostavlja ocenu i komentar, dok operater dobija uvid u rezultate i statistiku. Sistem ažurira podatke o učinku i vezuje ocene za konkretnu turu i vodiča. U slučaju prijavljenih problema, aktivira se administrativni postupak.

\begin{itemize}
    \item Osnovni tok: završetak ture -> ocena/recenzija -> ažuriranje statistike.
    \item Alternativni tok: prijava problema -> otvaranje slučaja -> obrada.
\end{itemize}

\begin{figure}[H]
    \centering
    \includegraphics[width=0.95\textwidth]{Slike/seq_post_tour.png}
    \caption{Dijagram sekvenci: Aktivnosti nakon ture}
    \label{fig:seq_post_tour}
\end{figure}
