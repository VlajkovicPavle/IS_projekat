DFD dijagrami prikazuju tok podataka kroz sistem. Nivo 0 predstavlja sistem kao jedinstvenu celinu u odnosu na spoljne aktere, dok nivo 1 razlaže proces na ključne podsisteme i tokove podataka. Na taj način se vidi kako informacije prolaze kroz pretragu, rezervaciju, upravljanje turama i evaluaciju.

DFD je koristan za razumevanje gde se podaci generišu, gde se čuvaju i koje komponente ih koriste. Takođe ukazuje na kritične tačke validacije i autorizacije, kao i na mesta gde je neophodna sinhronizacija između aktera i sistema.

\subsection*{Nivo 0}
Prikazani su glavni akteri (korisnik, vodič, operater i administrator) i njihova razmena podataka sa sistemom. Ovaj nivo naglašava osnovne ulaze i izlaze, bez detaljnog razlaganja internih procesa.

\begin{itemize}
    \item Ulazi: pretrage, zahtevi za rezervaciju, prijave i ocene.
    \item Izlazi: potvrde, obaveštenja, statusi tura i izveštaji.
\end{itemize}

\begin{figure}[H]
    \centering
    \includegraphics[width=0.95\textwidth]{Slike/dfd_level_0.png}
    \caption{DFD nivo 0}
    \label{fig:dfd_level_0}
\end{figure}

\subsection*{Nivo 1}
Nivo 1 razlaže sistem na logičke celine: pretraga i pregled tura, rezervacije, upravljanje turama i post-tur aktivnosti. Prikazani su tokovi podataka između podsistema i skladišta podataka.

\begin{itemize}
    \item Pretraga i pregled: rad sa katalogom tura i filtrima.
    \item Rezervacije: provera kapaciteta, kreiranje i otkazivanje.
    \item Upravljanje turama: kreiranje, dodela vodiča i ažuriranje termina.
    \item Post-tur aktivnosti: ocene, recenzije i analitika.
\end{itemize}

\begin{figure}[H]
    \centering
    \includegraphics[width=0.95\textwidth]{Slike/dfd_level_1.png}
    \caption{DFD nivo 1}
    \label{fig:dfd_level_1}
\end{figure}
