Informacioni sistem se razvija za organizaciju planinarskih tura i podršku celokupnom procesu, od pretrage i prijave do realizacije ture i evaluacije nakon završetka. Fokus je na objedinjavanju komunikacije između korisnika, vodiča i turističkih operatera, uz centralizovano upravljanje terminima, kapacitetima i rezervacijama. Sistem rešava česte probleme u planiranju tura: nedovoljnu dostupnost informacija, neusklađenost termina i otežanu koordinaciju učesnika.

Proces počinje pretragom dostupnih tura, pregledom detalja i odabirom odgovarajuće ponude. Nakon rezervacije, korisnik dobija jasne informacije o terminu, težini ture i potrebnoj opremi, kao i kanal komunikacije sa vodičem. Tokom realizacije ture, sistem omogućava praćenje ključnih informacija i ažuriranje statusa. Po završetku, korisnik ostavlja ocenu i komentar, dok operater i vodič dobijaju povratnu informaciju o kvalitetu usluge.

Predlog informacionog sistema izrađen je kao projekat u okviru predmeta „Informacioni sistemi“. Zbog obima i rokova, dokumentacija obuhvata ključne funkcionalnosti i modele neophodne za funkcionisanje platforme, uz smernice za dalji razvoj.

\subsection{Učesnici u sistemu}
Osnovnu podelu čine registrovani korisnici, vodiči i turistički operateri, uz administrativni sloj. Uloge su:
\begin{itemize}
    \item Korisnici – pretražuju i rezervišu ture, prate informacije i ostavljaju povratne ocene.
    \item Vodiči – realizuju ture, upravljaju dostupnošću, komuniciraju sa učesnicima i evidentiraju status.
    \item Turistički operateri – kreiraju ture, dodeljuju vodiče, upravljaju kapacitetima i analiziraju učinak.
    \item Administratori – nadgledaju sistem, odobravaju registracije i rešavaju sporove.
\end{itemize}

\subsection{Korišćeni dijagrami i alati}
Tokom izrade dokumentacije korišćeni su:
\begin{itemize}
    \item UML dijagrami (slučajevi upotrebe, sekvence, klasni dijagram).
    \item DFD dijagrami za tokove podataka.
    \item BPMN dijagrami za poslovne procese.
    \item C4 model za prikaz arhitekture sistema.
    \item Skice korisničkog interfejsa (prototipi korisničkog interfejsa).
\end{itemize}

Za izradu UML i BPMN dijagrama korišćen je Visual Paradigm. Skice interfejsa izrađene su u alatu Diagrams.net, dok je arhitektura modelovana C4 pristupom.
